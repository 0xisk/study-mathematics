%%%%%%%%%%%%%%%%%%%%%%%%%%%%%%%%%%%%%%%%%%%%%%%%%%%%%%%%%%%%%%%%%%%%%%%%%%%%%%%%
%2345678901234567890123456789012345678901234567890123456789012345678901234567890
%        1         2         3         4         5         6         7         8
 
\documentclass[letterpaper, 10 pt, conference]{ieeeconf}  % Comment this line out
                                                          % if you need a4paper
%\documentclass[a4paper, 10pt, conference]{ieeeconf}      % Use this line for a4
                                                          % paper
% \documentclass{article} 

\usepackage[english]{babel}
\usepackage[utf8]{inputenc}
\usepackage{amsmath}
\usepackage{blindtext}
\usepackage{scrextend}
\usepackage{fontawesome5}
\usepackage{graphicx}
\usepackage{hyperref}
\usepackage{amssymb}
\usepackage{upgreek}
\usepackage{listings}
\usepackage{amsmath}
\usepackage{amsthm}
\usepackage[export]{adjustbox}
\usepackage[colorinlistoftodos]{todonotes}
\IEEEoverridecommandlockouts                              
\overrideIEEEmargins

\definecolor{codegreen}{rgb}{0,0.6,0}
\definecolor{codegray}{rgb}{0.5,0.5,0.5}
\definecolor{codepurple}{rgb}{0.58,0,0.82}
\definecolor{backcolour}{rgb}{0.95,0.95,0.92}

\lstdefinestyle{mystyle}{
    backgroundcolor=\color{backcolour},   
    commentstyle=\color{codegreen},
    keywordstyle=\color{magenta},
    numberstyle=\tiny\color{codegray},
    stringstyle=\color{codepurple},
    basicstyle=\ttfamily\footnotesize,
    breakatwhitespace=false,         
    breaklines=true,                 
    captionpos=b,                    
    keepspaces=true,                 
    numbers=left,                    
    numbersep=5pt,                  
    showspaces=false,                
    showstringspaces=false,
    showtabs=false,                  
    tabsize=2
}

\lstset{style=mystyle}

\title{\LARGE \bf
ZKU – Cohort 4 (Jul-Aug 2022)\\Week 1: Introduction to ZKP\\Assignment \sharp 1}

% \author{Iskander Andrews\\\faIcon{discord} Isk#0996}
\author{Iskander Andrews$^{1}$% <-this % stops a space
\thanks{\faIcon[regular]{envelope} \tt\small iskander.s.andrews@gmail.com}
\thanks{\faIcon{discord} \tt\small Isk#0996}
\thanks{\faIcon{github} \tt\small iskdrews}}

\begin{document}



\maketitle
\thispagestyle{empty}
\pagestyle{empty}


%%%%%%%%%%%%%%%%%%%%%%%%%%%%%%%%%%%%%%%%%%%%%%%%%%%%%%%%%%%%%%%%%%%%%%%%%%%%%%%%
\begin{abstract}

This document is for answering ZKUC04, Week-1, Introduction to ZKP Assignment.
It consists of three main parts.
\begin{labeling}{alligator}
\item [\textbf{Part 1:}] Theoretical background of zk-SNARKs and zk-STARKS.
\item [\textbf{Part 2:}] Getting started with circom and snarkjs.
\item [\textbf{Part 3:}] Reading and designing circuits with Circom.
\end{labeling}

The total number of questions and points is $13$, including the bonus questions. 
\end{abstract}

\noindent\rule{8cm}{0.4pt}

\section{\textbf{Problems}}
\subsection{\textbf{\underline{Question (1)} Prove the following:}}
\subsubsection{\textbf{If $ax = a$ for some number $a \neq 0$, then $x = 1$.}}

\renewcommand\qedsymbol{$\blacksquare$}

\begin{proof}
\begin{align}
a.x& = a && \text{}\\
a^{-1}.a.x& = a.a^{-1} && \text{\tiny{From P(7) Multiplication Inverse}}\\
1.x& = 1 && \text{}\\
x& = 1 && 
\end{align}

\begin{flushright}
$\blacksquare$
\end{flushright}
\end{proof}

\subsubsection{$\boldsymbol{x^2 - y^2 = (x - y)(x + y)}$}

\begin{proof}
\begin{align}
     x^2 - y^2&= (x - y)(x + y) && \\
     &=[x + (-y)](x + y) && \text{\tiny{From P(9) Distributive Law}} \\ 
     &=x(x + y) + (-y)(x +y) &&\\
     &=x(x + y) - [y(x +y)] &&\\
     &=x^2 + xy - [yx + y^2] && \text{\tiny{From P(8) Commutative law for Multiplication}}\\
     &=x^2 + xy - xy - y^2 && \text{\tiny{From P(2) Addition Inverses}}\\
     &=x^2 + 0 - y^2 &&\\
     &=x^2 - y^2 &&
\end{align}
\begin{flushright}
$\blacksquare$
\end{flushright}
\end{proof}


\subsubsection{\textbf{If $x^2 = y^2$, then $x = y$ or $x = -y$.}}


\begin{proof}
\begin{align}
    x^2 &= y^2 && \\
    xx &= yy && \\
    x^{-1}.x.x &= x^{-1}.y.y &&\\
    [x^{-1}.x].x &= x^{-1}.y.y && \\
    [1].x & = \frac{y.y}{x} && \\
    x & = \frac{y.y}{x} && \\
    [y^{-1}].x & = \frac{y.y}{x.y^{-1}} && \\
    \frac{x}{y}& = \frac{y}{y} . \frac{y}{x} && \\
    \frac{x}{y}& = 1 . \frac{y}{x} && \\
    \frac{x}{y}& = \frac{y}{x} && \\
\end{align}
\end{proof}




\renewcommand\qedsymbol{QED}

\noindent\rule{8cm}{0.4pt}
\newpage

\section*{ACKNOWLEDGMENT}
I would like to thank so much Heather @Giveth who told me to register in this great course, and would like to thank hadzija#0842 and cs#6500 for their great help and support. I am really thankful to them and to everyone working behind the scenes to create, maintain, review, manage, and produce this great ZKP course, I see great efforts. 

\noindent\rule{8cm}{0.4pt}
\begin{thebibliography}{99}

\bibitem{c1} Mattison Asher, Coogan Brennan, Zero-Knowledge Proofs: STARKs vs SNARKs, May 18, 2021. \href{https://consensys.net/blog/blockchain-explained/zero-knowledge-proofs-starks-vs-snarks/}{\underline{link}}
\bibitem{c2} EthHub, ZK-STARKs. \href{https://docs.ethhub.io/ethereum-roadmap/layer-2-scaling/zk-starks/}{\underline{link}}
\bibitem{c3} github.com/ebfull/powersoftau. \href{https://github.com/ebfull/powersoftau}{\underline{link}}
\bibitem{c4} Koh Wei Jie, Announcing the Perpetual Powers of Tau Ceremony to benefit all zk-SNARK projects, Sep 11, 2019. \href{https://medium.com/coinmonks/announcing-the-perpetual-powers-of-tau-ceremony-to-benefit-all-zk-snark-projects-c3da86af8377#:~:text=The%20Powers%20of%20Tau%20ceremony,protocol%20can%20be%20publicly%20verified.&text=Nevertheless%2C%20each%20ceremony%20takes%20time%20and%20is%20tedious%20to%20coordinate.}{\underline{link}}
\bibitem{c5} Polygon Hermez, Hermez Zero-Knowledge Proofs, 2020. \href{https://blog.hermez.io/hermez-zero-knowledge-proofs/}{\underline{link}}
\bibitem{c6} Matthew Finestone, Loopring Begins zkSNARK Trusted Setup Multi-Party Computation Ceremony, Nov 10, 2019 \href{https://medium.com/coinmonks/loopring-starts-zksnark-trusted-setup-multi-party-computation-ceremony-cd1b98113773#:~:text=There%20are%20two%20phases%20for,circuit%20in%20the%20Loopring%20protocol.}{\underline{link}}
\bibitem{c7} Circom official documentations. \href{https://docs.circom.io/circom-language/constraint-generation/}{\underline{link}}
\bibitem{c8} Snarkjs official documentation \href{https://github.com/iden3/snarkjs/blob/master/README.md}{\underline{link}}
\end{thebibliography}

\end{document}
