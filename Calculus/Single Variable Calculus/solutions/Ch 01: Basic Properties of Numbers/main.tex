%%%%%%%%%%%%%%%%%%%%%%%%%%%%%%%%%%%%%%%%%%%%%%%%%%%%%%%%%%%%%%%%%%%%%%%%%%%%%%%%
%2345678901234567890123456789012345678901234567890123456789012345678901234567890
%        1         2         3         4         5         6         7         8
 
\documentclass[letterpaper, 10 pt, conference]{ieeeconf}  % Comment this line out
                                                          % if you need a4paper
%\documentclass[a4paper, 10pt, conference]{ieeeconf}      % Use this line for a4
                                                          % paper
% \documentclass{article} 

\usepackage[english]{babel}
\usepackage[utf8]{inputenc}
\usepackage{amsmath}
\usepackage{blindtext}
\usepackage{scrextend}
\usepackage{fontawesome5}
\usepackage{enumitem}
\usepackage{graphicx}
\usepackage{hyperref}
\usepackage{amssymb}
\usepackage{upgreek}
\usepackage{listings}
\usepackage{amsmath}
\usepackage{amsthm}
\usepackage[export]{adjustbox}
\usepackage[colorinlistoftodos]{todonotes}
\IEEEoverridecommandlockouts                              
\overrideIEEEmargins

\definecolor{codegreen}{rgb}{0,0.6,0}
\definecolor{codegray}{rgb}{0.5,0.5,0.5}
\definecolor{codepurple}{rgb}{0.58,0,0.82}
\definecolor{backcolour}{rgb}{0.95,0.95,0.92}

% Romainan number 

\lstdefinestyle{mystyle}{
    backgroundcolor=\color{backcolour},   
    commentstyle=\color{codegreen},
    keywordstyle=\color{magenta},
    numberstyle=\tiny\color{codegray},
    stringstyle=\color{codepurple},
    basicstyle=\ttfamily\footnotesize,
    breakatwhitespace=false,         
    breaklines=true,                 
    captionpos=b,                    
    keepspaces=true,                 
    numbers=left,                    
    numbersep=5pt,                  
    showspaces=false,                
    showstringspaces=false,
    showtabs=false,                  
    tabsize=2
}

\lstset{style=mystyle}

\title{\LARGE \bf
Single Variable Calculus – Micheal Spivak\\3\textsuperscript{rd} Edition\\Part \rom{1}: Prologue\\Ch 01: Basic Properties of Numbers}

% \author{Iskander Andrews\\\faIcon{discord} Isk#0996}
\author{Iskander Andrews$^{1}$% <-this % stops a space
\thanks{\noindent\rule{3cm}{0.4pt}}
\thanks{\faIcon{discord} \tt\small Isk#0996}
\thanks{\faIcon{twitter} \tt\small \href{https://twitter.com/iskdrews}{twitter.com/iskdrews}}
\thanks{\faIcon{github} \tt\small \href{https://github.com/iskdrews}{github.com/iskdrews}}
\thanks{\faIcon{linkedin} \tt\small \href{https://www.linkedin.com/in/iskander-andrews-99638313a/}{in/iskander-andrews}}
\thanks{\faIcon[regular]{envelope} \tt\small iskander.s.andrews@gmail.com}
}
\begin{document}



\maketitle
\thispagestyle{empty}
\pagestyle{empty}


%%%%%%%%%%%%%%%%%%%%%%%%%%%%%%%%%%%%%%%%%%%%%%%%%%%%%%%%%%%%%%%%%%%%%%%%%%%%%%%%
\begin{abstract}

This document is for answering Part \rom{1}: Prologue, Ch 01: "Basic Properties of Numbers" problems.
% It consists of three main parts.
% \begin{labeling}{alligator}
% \item [\textbf{Part 1:}] Theoretical background of zk-SNARKs and zk-STARKS.
% \item [\textbf{Part 2:}] Getting started with circom and snarkjs.
% \item [\textbf{Part 3:}] Reading and designing circuits with Circom.
% \end{labeling}

% The total number of questions and points is $13$, including the bonus questions. 
\end{abstract}

\noindent\rule{8cm}{0.4pt}

\section{\textbf{Problems}}

%%%%%%%%%%%%%%%%%%%%%%%%%%%%%%%%%%%%%%%%%%%%%%%%%%%%%%
%%%%%%%%%%%%%%%%%%% Problem (1) %%%%%%%%%%%%%%%%%%%%%%
%%%%%%%%%%%%%%%%%%%%%%%%%%%%%%%%%%%%%%%%%%%%%%%%%%%%%%

\subsection{\textbf{\underline{Problem (1)} Prove the following:}}
\subsubsection{\textbf{If $ax = a$ for some number $a \neq 0$, then $x = 1$.}}

\renewcommand\qedsymbol{$\blacksquare$}

\begin{proof}
\begin{align}
a.x& = a && \text{}\\
a^{-1}.a.x& = a.a^{-1} && \text{\tiny{From P(7) Multiplication Inverse}}\\
1.x& = 1 && \text{}\\
x& = 1 && 
\end{align}

\begin{flushright}
$\blacksquare$
\end{flushright}
\end{proof}

\subsubsection{$\boldsymbol{x^2 - y^2 = (x - y)(x + y)}$}

\begin{proof}
\begin{align}
     x^2 - y^2&= (x - y)(x + y) && \\
     &=[x + (-y)](x + y) && \text{\tiny{From P(9) Distributive Law}} \\ 
     &=x(x + y) + (-y)(x +y) &&\\
     &=x(x + y) - [y(x +y)] &&\\
     &=x^2 + xy - [yx + y^2] && \text{\tiny{From P(8) Commutative law for Multiplication}}\\
     &=x^2 + xy - xy - y^2 && \text{\tiny{From P(2) Addition Inverses}}\\
     &=x^2 + 0 - y^2 &&\\
     &=x^2 - y^2 &&
\end{align}
\begin{flushright}
$\blacksquare$
\end{flushright}
\end{proof}


\subsubsection{\textbf{If $x^2 = y^2$, then $x = y$ or $x = -y$.}}

\begin{proof}
\begin{align}
    
\end{align}
\begin{flushright}
$\blacksquare$
\end{flushright}
\end{proof}

\subsubsection{\textbf{$x^3 = y^3$, then $(z - y)(x^2 + xy +y^2))$.}}

\begin{proof}
\begin{align}
    
\end{align}
\begin{flushright}
$\blacksquare$
\end{flushright}
\end{proof}

\subsubsection{\textbf{$x^n = y^n$, then \\ $(z - y)(x^{n-1} + x^{n-2}y + \dots + xy^{n-2} + y^{n-1})$.}}

\begin{proof}
\begin{align}
    
\end{align}
\begin{flushright}
$\blacksquare$
\end{flushright}
\end{proof}

\subsubsection{\textbf{$x^3 = y^3$, then $(z - y)(x^2 - xy +y^2))$. \\ \small{There is a particularly easy way to do this, using (4), and it will show you how to find a factorization for $x^n + y^n$} whenever $n$ is odd.}}

\begin{proof}
\begin{align}
    
\end{align}
\begin{flushright}
$\blacksquare$
\end{flushright}
\end{proof}

\noindent\rule{8cm}{0.4pt}

%%%%%%%%%%%%%%%%%%%%%%%%%%%%%%%%%%%%%%%%%%%%%%%%%%%%%%
%%%%%%%%%%%%%%%%%%% Problem (2) %%%%%%%%%%%%%%%%%%%%%%
%%%%%%%%%%%%%%%%%%%%%%%%%%%%%%%%%%%%%%%%%%%%%%%%%%%%%%
\subsection{\textbf{\underline{Problem (2)} What is wrong with the following "Proof"? \\ Let $x = y$. Then:}}

\begin{proof}
\begin{align}
    &\text{Wrong Proof: }&&\\
    x^2 &= xy&&\\
    x^2 - y^2 &= xy - y^2&&\\
    (x + y)(x - y) &= y(x - y)&&\\
    x + y&= y&&\\
    2y &= y &&\\
    2& = 1&&\\
    &\text{Correnct Proof: }&&\\
\end{align}
\begin{flushright}
$\blacksquare$
\end{flushright}
\end{proof}

\noindent\rule{8cm}{0.4pt}

%%%%%%%%%%%%%%%%%%%%%%%%%%%%%%%%%%%%%%%%%%%%%%%%%%%%%%
%%%%%%%%%%%%%%%%%%% Problem (3) %%%%%%%%%%%%%%%%%%%%%%
%%%%%%%%%%%%%%%%%%%%%%%%%%%%%%%%%%%%%%%%%%%%%%%%%%%%%%
\subsection{\textbf{\underline{Problem (3)} Prove the following: }}

\subsubsection{\textbf{$\frac{a}{b} = \frac{ac}{bc}$, if $b, c \neq 0$.}}
\begin{proof}
\begin{align}
    
\end{align}
\begin{flushright}
$\blacksquare$
\end{flushright}
\end{proof}

\subsubsection{\textbf{$\frac{a}{b} + \frac{c}{d} = \frac{ad + bc}{bd}$, if $b, d \neq 0$.}}
\begin{proof}
\begin{align}
    
\end{align}
\begin{flushright}
$\blacksquare$
\end{flushright}
\end{proof}

\subsubsection{\textbf{$(ab)^{-1} = a^{-1}b^{-1}$, if $a, b \neq 0$. \\ \small{To do this you must remember the defining property of $(ab)^{-1}$}}}
\begin{proof}
\begin{align}
    
\end{align}
\begin{flushright}
$\blacksquare$
\end{flushright}
\end{proof}

\subsubsection{\textbf{$\frac{a}{b} . \frac{c}{d} = \frac{ac}{db}$, if $b, d \neq 0$.}}
\begin{proof}
\begin{align}
    
\end{align}
\begin{flushright}
$\blacksquare$
\end{flushright}
\end{proof}

\subsubsection{\textbf{$\frac{a}{b} / \frac{c}{d} = \frac{ad}{bc}$, if $b, c, d \neq 0$.}}
\begin{proof}
\begin{align}
    
\end{align}
\begin{flushright}
$\blacksquare$
\end{flushright}
\end{proof}

\subsubsection{\textbf{If $b, d \neq 0$, then $\frac{a}{b} = \frac{c}{d}$ if and only if $ad = bc$. \\ Also determine when $\frac{a}{b} = \frac{b}{a}$.}}
\begin{proof}
\begin{align}
    
\end{align}
\begin{flushright}
$\blacksquare$
\end{flushright}
\end{proof}

\noindent\rule{8cm}{0.4pt}

%%%%%%%%%%%%%%%%%%%%%%%%%%%%%%%%%%%%%%%%%%%%%%%%%%%%%%
%%%%%%%%%%%%%%%%%%% Problem (4) %%%%%%%%%%%%%%%%%%%%%%
%%%%%%%%%%%%%%%%%%%%%%%%%%%%%%%%%%%%%%%%%%%%%%%%%%%%%%
\subsection{\textbf{\underline{Problem (4)} Find all numbers $x$ for which: }}

\subsubsection{\textbf{$4 - x < 3 - 2x$.}}
\begin{proof}
\begin{align}
    
\end{align}
\begin{flushright}
$\blacksquare$
\end{flushright}
\end{proof}

\subsubsection{\textbf{$5 - x^2 < 8$.}}
\begin{proof}
\begin{align}
    
\end{align}
\begin{flushright}
$\blacksquare$
\end{flushright}
\end{proof}

\subsubsection{\textbf{$5 - x^2 < -2$.}}
\begin{proof}
\begin{align}
    
\end{align}
\begin{flushright}
$\blacksquare$
\end{flushright}
\end{proof}

\subsubsection{\textbf{$(x - 1)(x -3) > 0$. \\ (Which is a product of two numbers positive?)}}
\begin{proof}
\begin{align}
    
\end{align}
\begin{flushright}
$\blacksquare$
\end{flushright}
\end{proof}

\subsubsection{\textbf{$x^2 - 2x + 2 > 0$.}}
\begin{proof}
\begin{align}
    
\end{align}
\begin{flushright}
$\blacksquare$
\end{flushright}
\end{proof}

\subsubsection{\textbf{$x^2 + x + 1 > 2$.}}
\begin{proof}
\begin{align}
    
\end{align}
\begin{flushright}
$\blacksquare$
\end{flushright}
\end{proof}

\subsubsection{\textbf{$x^2 - x + 10 > 16$.}}
\begin{proof}
\begin{align}
    
\end{align}
\begin{flushright}
$\blacksquare$
\end{flushright}
\end{proof}

\subsubsection{\textbf{$x^2 + x + 1 > 0$.}}
\begin{proof}
\begin{align}
    
\end{align}
\begin{flushright}
$\blacksquare$
\end{flushright}
\end{proof}

\subsubsection{\textbf{$(x - \pi)(x + 5)(x - 3) > 0$.}}
\begin{proof}
\begin{align}
    
\end{align}
\begin{flushright}
$\blacksquare$
\end{flushright}
\end{proof}

\subsubsection{\textbf{$(x - \sqrt[3]{2})(x - \sqrt{2})$}}
\begin{proof}
\begin{align}
    
\end{align}
\begin{flushright}
$\blacksquare$
\end{flushright}
\end{proof}

\subsubsection{\textbf{$2^x < 8$.}}
\begin{proof}
\begin{align}
    
\end{align}
\begin{flushright}
$\blacksquare$
\end{flushright}
\end{proof}

\subsubsection{\textbf{$x + 3^x < 4$.}}
\begin{proof}
\begin{align}
    
\end{align}
\begin{flushright}
$\blacksquare$
\end{flushright}
\end{proof}

\subsubsection{\textbf{$\frac{1}{x} + \frac{1}{1 - x} > 0$.}}
\begin{proof}
\begin{align}
    
\end{align}
\begin{flushright}
$\blacksquare$
\end{flushright}
\end{proof}

\subsubsection{\textbf{$\frac{x -1}{x + 1} > 0$.}}
\begin{proof}
\begin{align}
    
\end{align}
\begin{flushright}
$\blacksquare$
\end{flushright}
\end{proof}

\noindent\rule{8cm}{0.4pt}

%%%%%%%%%%%%%%%%%%%%%%%%%%%%%%%%%%%%%%%%%%%%%%%%%%%%%%
%%%%%%%%%%%%%%%%%%% Problem (5) %%%%%%%%%%%%%%%%%%%%%%
%%%%%%%%%%%%%%%%%%%%%%%%%%%%%%%%%%%%%%%%%%%%%%%%%%%%%%
\begin{figure}[thpb]
      \centering
      \framebox{\parbox{2.1in}{Problem (5) answers are from Ref \cite{c1}}}
\end{figure}
\subsection{\textbf{\underline{Problem (5)} Prove the following:}}

\subsubsection{\textbf{If $a < b$ and $c < d$, then $a + c < b + d$.}}

\begin{proof}
\begin{align}
    &a < b \rightarrow b - a > 0 &&\\
    &c < d \rightarrow d - c > 0 &&\\
    &\text{\small{Since $b -a$, and $d - c $ are both positive,}} &&\\
    &b + d - (c + a) > 0 && \text{\tiny{ By P(11) Closure under addition}} \\
    \text{Therefore: } &a + c < b + d &&
\end{align}
\begin{flushright}
$\blacksquare$
\end{flushright}
\end{proof}


\subsubsection{\textbf{If $a < b$ and $c > d$, then $a - c < b - d$.}}

\begin{proof}
\begin{align}
    &a < b \rightarrow b - a > 0 &&\\
    &c > d \rightarrow c - d > 0 &&\\
    \text{Since: }&\text{$b - a$, and $c - d$ are both positive,} &&\\
    &b - a + (c - d) > 0 && \\
    &-a + c + b - d > 0 &&\\
    &(-a + c) + (b - d) > 0 &&\\
    \text{Therefore: } & b - d > - [-a + c]&&\\
    & a - c < b - d&&
\end{align}
\begin{flushright}
$\blacksquare$
\end{flushright}
\end{proof}

\subsubsection{\textbf{If $a < b$, then $-b < -a$.}}

\begin{proof}
\begin{align}
    &a < b \rightarrow b - a > 0 &&\\
    \text{Then } &b - a &&\text{ is positive.} \\
    \text{Then } &-(b - a) &&\\
    \text{Then } &a - b&& \text{ is negative.} \\
    \text{Since } & a - b < 0 && \\
    \text{Therefore} & -b < -a &&
\end{align}
\begin{flushright}
$\blacksquare$
\end{flushright}
\end{proof}

\subsubsection{\textbf{If $a < b$ and $c > 0$, then $a.c < b.c$.}}
\begin{proof}
\begin{align}
    &a < b \rightarrow b - a > 0 &&\\
    \text{Since: } & c > 0 && \\
    & c.(b - a) > 0.c && \\
    & c.(b - a) > 0 && \\
    & c.b - c.a > 0 && \text{\tiny{By P(9) Distributive law}}\\
    & c.b - c.a + c.a > 0 + c.a && \text{\tiny{By P(3) Additive inverse.}}\\
    & c.b > c.a && \text{\tiny{By P(2) Additive identity.}}\\
    &b.c > a.c && \text{\tiny{By P(8) Commutative law for multiplication.}} \\
    \text{Therefore: } & a.c < b.c&&
\end{align}
\begin{flushright}
$\blacksquare$
\end{flushright}
\end{proof}

\subsubsection{\textbf{If $a < b$ and $c < 0$, then $a.c > b.c$}}

\begin{proof}
\begin{align}
    &a < b \rightarrow b - a > 0 &&\\
    &\becuase c < 0 && \\
    & c.(b - a) > 0.c && \text{\tiny{Multiply both sides with $c$}} \\
    & c.b - a.c > 0 && \text{\tiny{By P(9) Distributive law.}}\\
    & c.b - a.c + a.c > 0 + a.c && \text{\tiny{By P(3) Additive inverses.}}\\
    \text{Therefore: }& c.b > a.c \equiv a.c < b.c &&
\end{align}
\begin{flushright}
$\blacksquare$
\end{flushright}
\end{proof}

\subsubsection{\textbf{If $a > 1$, then $a^2 > a$.}}

\begin{proof}
\begin{align}
    &\because a > 1 && \text{\tiny{Multiply both sides with $a$}} \\
    &\therefore a.a = a^2 > a &&\\
    &\because a > 1 \therefore a > 0 &&  \\
    &\because c > b \text{ and }  d > 0 \therefore c.d > b.d && \text{\tiny{Proven at P(5.4)}} \\
    \text{let: } & d = c = a \text{ and } b = 1 && \\
    & \therefore a.a > a.1 \rightarrow a^2 > a&&
\end{align}
\begin{flushright}
$\blacksquare$
\end{flushright}
\end{proof}

\subsubsection{\textbf{If $0 < a < 1$, then $a^2 < a$.}}

\begin{proof}
\begin{align}
    &\because a < 1 && \text{\tiny{Multiply both sides with $a$}}\\
    &\therefore aa < a \rightarrow a^2 < a &&\\
    &\because a < 1 \because a > 0 &&  \\
    &\because a < b \text{ and }  c > 0 \therefore a.c < b.c && \text{\tiny{Proven at P(5.4)}} \\
    \text{let: } & a = c = a \text{ and } b = 1 && \text{\tiny{By Substitute}.}\\
    & \therefore a.a < 1.a \rightarrow a^2 > a&&
\end{align}
\begin{flushright}
$\blacksquare$
\end{flushright}
\end{proof}

\subsubsection{\textbf{If $0 \leq  a < b$, and $0 \leq c < d$, then $ac < bd$.}}


\begin{proof}
\begin{align}
    &\text{Case (1): } &&\\
    \text{let: } & a, c \neq 0 &&\\
    &\because a < b \therefore ac < bc && \text{\tiny{Multiply $a < b$ by $c$ on both sides.}} \\
    &\because c < d \therefore bc < bd && \text{\tiny{Multiply $c < d$ by $b$ on both sides.}} \\
    &\because ac < bc \text{ and } bc < bd && \text{\tiny{By the transitive property}} \\ 
    &\therefore ac < bd && \\\\
    &\text{Case (2): } &&\\
    \text{let: } & a = 0 \text { or } c = 0 \text{ or both. }&&\\
    &\therefore ac = 0&&\\
    &\because 0 \leq a < b \text{ and } 0 \leq c < d && \\
    &\because b > 0 \text{ and } d > 0 && \text{\tiny{By closure under multiplication of inequalities.}}\\
    &\therefore bd > 0 = ac &&\\
    &\therefore ac < bd.&&
\end{align}
\begin{flushright}
$\blacksquare$
\end{flushright}
\end{proof}

\subsubsection{\textbf{If $0 \leq a < b$, then $a^2 < b^2$, then $a^2 < b^2$, (Use 8.}}

\begin{proof}
\begin{align}
    &\because \text{In Q(5.8) says if:} && \\
    &0 \leq a < b, \text{ and } 0 \leq c < d && \\ 
    &\therefore ac < bd && \\ 
    &\text{Let: } a = c, \text{ and } d = b&&\\
    &\therefore a^2 < b^2 &&
\end{align}
\begin{flushright}
$\blacksquare$
\end{flushright}
\end{proof}

\subsubsection{\textbf{If $a, b \geqslant  0$, and $a^2 < b^2$, then $a < b$. (Use 9 backwards).}}

\begin{proof}
\begin{align}
    &\text{Assume } a < b \text{ were false, so } a = b \text{ or } a > b && \\
    &\text{Case (1): } && \\
    &\becuase a = b && \\
    &\therefore a^2 = b^2 && \\
    &\therefore \text{That contradicts our given that: } a^2 < b^2 && \\
    &\text{Case (2): } && \\
    &\because a > b, \text{ and } b \geq 0, &&\\
    &\therefore \text{By part 9, } a^2 > b^2 &&\\
    &\therefore \text{That contradicts our given that: } a^2 < b^2 &&\\ 
    &\text{Conclusion: }&&\\
    &\because \text{Assumption that } a < b \text{ is false can only lead to contradictions.} &&\\
    &\therefore \text{We can assert that: } a < b &&
\end{align}
\begin{flushright}
$\blacksquare$
\end{flushright}
\end{proof}

\noindent\rule{8cm}{0.4pt}
%%%%%%%%%%%%%%%%%%%%%%%%%%%%%%%%%%%%%%%%%%%%%%%%%%%%%%
%%%%%%%%%%%%%%%%%%% Problem (6) %%%%%%%%%%%%%%%%%%%%%%
%%%%%%%%%%%%%%%%%%%%%%%%%%%%%%%%%%%%%%%%%%%%%%%%%%%%%%
\begin{figure}[thpb]
      \centering
      \framebox{\parbox{2.1in}{Problem (6) answers are from Ref \cite{c1}}}
\end{figure}
\subsection{\textbf{\underline{Problem (6)} Prove the following:}}

\subsubsection{\textbf{If $0 \leq x < y$, then $x^n < y^n, n = 1, 2, 3, \ldots $}}
\begin{proof}
\begin{align}
    
\end{align}
\begin{flushright}
$\blacksquare$
\end{flushright}
\end{proof}

\subsubsection{\textbf{If $x < y$ and $n$ is odd, then $x^n < y^n$.}}
\begin{proof}
\begin{align}
    
\end{align}
\begin{flushright}
$\blacksquare$
\end{flushright}
\end{proof}

\subsubsection{\textbf{If $x^n = y^n$ and $n$ is odd, then $x = y$.}}
\begin{proof}
\begin{align}
    
\end{align}
\begin{flushright}
$\blacksquare$
\end{flushright}
\end{proof}


\subsubsection{\textbf{If $x^n = y^n$ and $n$ is even, then $x = y$ or $x = -y$.}}
\begin{proof}
\begin{align}
    
\end{align}
\begin{flushright}
$\blacksquare$
\end{flushright}
\end{proof}


\noindent\rule{8cm}{0.4pt}

%%%%%%%%%%%%%%%%%%%%%%%%%%%%%%%%%%%%%%%%%%%%%%%%%%%%%%
%%%%%%%%%%%%%%%%%%% Problem (7) %%%%%%%%%%%%%%%%%%%%%%
%%%%%%%%%%%%%%%%%%%%%%%%%%%%%%%%%%%%%%%%%%%%%%%%%%%%%%
\subsection{\textbf{\underline{Problem (7)} Prove that:}}

\subsubsection{\textbf{If $0 < a < b$, then: $a < \sqrt{ab} < \frac{a + b}{2} < b$ \\ - Notice that the inequality $\sqrt{ab} \leq (a + b) / 2$ holds for all $a, b \geq 0$. \\ - A generalization of this fact occurs in Problem 2-22.}}
\begin{proof}
\begin{align}
    
\end{align}
\begin{flushright}
$\blacksquare$
\end{flushright}
\end{proof}

\noindent\rule{8cm}{0.4pt}
%%%%%%%%%%%%%%%%%%%%%%%%%%%%%%%%%%%%%%%%%%%%%%%%%%%%%%
%%%%%%%%%%%%%%%%%%% Problem (8) %%%%%%%%%%%%%%%%%%%%%%
%%%%%%%%%%%%%%%%%%%%%%%%%%%%%%%%%%%%%%%%%%%%%%%%%%%%%%
\subsection{\textbf{\underline{Problem (8)} Although the basic properties of inequalities were stated in terms of the collection $P$ of all positive numbers, and $<$ was defined in terms of $P$, this procedure can be reversed. \\ Suppose that $P10 - P12$ are replaced by:}}

\begin{description}[align=left,labelwidth=3cm]
\item [P\prime10] For any numbers $a$ and $b$ one, and only one, of the following holds: 
    \begin{enumerate}
        \item $a = b$
        \item $a < b$
        \item $b < a$
    \end{enumerate}
\item [P\prime11] For any numbers $a$, $b$, and $c$, if $a < b$ and $b < c$, then $a < c$.
\item [P\prime12] For any numbers $a$, $b$, and $c$, if $a < b$, then $a + c < b + c$
\item [P\prime13] For any numbers $a$, $b$, and $c$, if $a < b$ and $0 < c$, then $ac < bc$.
\end{description}
Show that $P10-P12$ can then be deduced as theorems.

\begin{proof}
\begin{align}
    
\end{align}
\begin{flushright}
$\blacksquare$
\end{flushright}
\end{proof}


\noindent\rule{8cm}{0.4pt}
%%%%%%%%%%%%%%%%%%%%%%%%%%%%%%%%%%%%%%%%%%%%%%%%%%%%%%
%%%%%%%%%%%%%%%%%%% Problem (9) %%%%%%%%%%%%%%%%%%%%%%
%%%%%%%%%%%%%%%%%%%%%%%%%%%%%%%%%%%%%%%%%%%%%%%%%%%%%%
\subsection{\textbf{\underline{Problem (9)} Express each of the following with at least one less pair of absolute value signs:}}

\subsubsection{\textbf{$\left\lvert \sqrt{2} + \sqrt{3} - \sqrt{5} + \sqrt{7} \right\rvert$}}
\begin{proof}
\begin{align}
    
\end{align}
\begin{flushright}
$\blacksquare$
\end{flushright}
\end{proof}

\subsubsection{\textbf{$\left\lvert 
        (\left\lvert a + b \right\rvert - \left\lvert a \right\rvert - \left\lvert b \right\rvert) 
    \right\rvert$}}
\begin{proof}
\begin{align}
    
\end{align}
\begin{flushright}
$\blacksquare$
\end{flushright}
\end{proof}

\subsubsection{\textbf{$\left\lvert (\left\lvert a + b \right\rvert + \left\lvert c \right\rvert - \left\lvert a + b + c \right\rvert) \right\rvert$}}
\begin{proof}
\begin{align}
    
\end{align}
\begin{flushright}
$\blacksquare$
\end{flushright}
\end{proof}

\subsubsection{\textbf{$\left\lvert x^2 - 2xy + y^2\right\rvert$}}
\begin{proof}
\begin{align}
    
\end{align}
\begin{flushright}
$\blacksquare$
\end{flushright}
\end{proof}

\subsubsection{\textbf{$\left\lvert (\left\lvert \sqrt{2} + \sqrt{3} \right\rvert - \left\lvert \sqrt{5} - \sqrt{7} \right\rvert) \right\rvert$}}
\begin{proof}
\begin{align}
    
\end{align}
\begin{flushright}
$\blacksquare$
\end{flushright}
\end{proof}

\noindent\rule{8cm}{0.4pt}
%%%%%%%%%%%%%%%%%%%%%%%%%%%%%%%%%%%%%%%%%%%%%%%%%%%%%%
%%%%%%%%%%%%%%%%%%% Problem (10) %%%%%%%%%%%%%%%%%%%%%%
%%%%%%%%%%%%%%%%%%%%%%%%%%%%%%%%%%%%%%%%%%%%%%%%%%%%%%
\subsection{\textbf{\underline{Problem (10)} Express each of the following without absolute value signs, treating various cases separately when necessary.}}

\subsubsection{\textbf{$\left\lvert a + b \right\rvert - \left\lvert b \right\rvert$}}
\begin{proof}
\begin{align}
    
\end{align}
\begin{flushright}
$\blacksquare$
\end{flushright}
\end{proof}

\subsubsection{\textbf{$\left\lvert (\left\lvert x \right\rvert - 1) \right\rvert$}}
\begin{proof}
\begin{align}
    
\end{align}
\begin{flushright}
$\blacksquare$
\end{flushright}
\end{proof}

\subsubsection{\textbf{$\left\lvert x \right\rvert - \left\lvert x^2 \right\rvert$}}
\begin{proof}
\begin{align}
    
\end{align}
\begin{flushright}
$\blacksquare$
\end{flushright}
\end{proof}

\subsubsection{\textbf{$a- \left\lvert (a - \left\lvert a \right\rvert)\right\rvert$}}
\begin{proof}
\begin{align}
    
\end{align}
\begin{flushright}
$\blacksquare$
\end{flushright}
\end{proof}

\noindent\rule{8cm}{0.4pt}
%%%%%%%%%%%%%%%%%%%%%%%%%%%%%%%%%%%%%%%%%%%%%%%%%%%%%%
%%%%%%%%%%%%%%%%%%% Problem (11) %%%%%%%%%%%%%%%%%%%%%%
%%%%%%%%%%%%%%%%%%%%%%%%%%%%%%%%%%%%%%%%%%%%%%%%%%%%%%
\subsection{\textbf{\underline{Problem (11)} Find all numbers $x$ for which:}}
\subsubsection{\textbf{$\left\lvert x - 3 \right\rvert = 8$.}}
\begin{proof}
\begin{align}
    
\end{align}
\begin{flushright}
$\blacksquare$
\end{flushright}
\end{proof}

\subsubsection{\textbf{$\left\lvert x -3 \right\rvert < 8$.}}
\begin{proof}
\begin{align}
    
\end{align}
\begin{flushright}
$\blacksquare$
\end{flushright}
\end{proof}

\subsubsection{\textbf{$\left\lvert x + 4 \right\rvert < 2$.}}
\begin{proof}
\begin{align}
    
\end{align}
\begin{flushright}
$\blacksquare$
\end{flushright}
\end{proof}

\subsubsection{\textbf{$\left\lvert x - 1 \right\rvert + \left\lvert x - 2 \right\rvert > 1$.}}
\begin{proof}
\begin{align}
    
\end{align}
\begin{flushright}
$\blacksquare$
\end{flushright}
\end{proof}

\subsubsection{\textbf{$\left\lvert x - 1 \right\rvert + \left\lvert x + 1 \right\rvert < 2$.}}
\begin{proof}
\begin{align}
    
\end{align}
\begin{flushright}
$\blacksquare$
\end{flushright}
\end{proof}

\subsubsection{\textbf{$\left\lvert x - 1 \right\rvert + \left\lvert x + 1 \right\rvert < 1$.}}
\begin{proof}
\begin{align}
    
\end{align}
\begin{flushright}
$\blacksquare$
\end{flushright}
\end{proof}

\subsubsection{\textbf{$\left\lvert x - 1 \right\rvert . \left\lvert x + 1 \right\rvert  = 0$.}}
\begin{proof}
\begin{align}
    
\end{align}
\begin{flushright}
$\blacksquare$
\end{flushright}
\end{proof}

\subsubsection{\textbf{$\left\lvert x - 1 \right\rvert . \left\lvert x + 2 \right\rvert = 3$.}}
\begin{proof}
\begin{align}
    
\end{align}
\begin{flushright}
$\blacksquare$
\end{flushright}
\end{proof}

\noindent\rule{8cm}{0.4pt}
%%%%%%%%%%%%%%%%%%%%%%%%%%%%%%%%%%%%%%%%%%%%%%%%%%%%%%
%%%%%%%%%%%%%%%%%%% Problem (12) %%%%%%%%%%%%%%%%%%%%%%
%%%%%%%%%%%%%%%%%%%%%%%%%%%%%%%%%%%%%%%%%%%%%%%%%%%%%%
\subsection{\textbf{\underline{Problem (12)} Prove the following:}}

\subsubsection{\textbf{$\left\lvert xy  \right\rvert = \left\lvert x \right\rvert . \left\lvert y \right\rvert$.}}
\begin{proof}
\begin{align}
    
\end{align}
\begin{flushright}
$\blacksquare$
\end{flushright}
\end{proof}

\subsubsection{\textbf{$\left\lvert \frac{1}{x} \right\rvert = \frac{1}{\left\lvert x \right\rvert}$, if $x \neq 0 $. (The best way to do this is to remember what $\left\lvert x^{-1}\right\rvert$ is.)}}
\begin{proof}
\begin{align}
    
\end{align}
\begin{flushright}
$\blacksquare$
\end{flushright}
\end{proof}

\subsubsection{\textbf{$\frac{\left\lvert x \right\rvert}{\left\lvert y \right\rvert} = \left\lvert \frac{x}{y} \right\rvert$, if $y \neq 0$.}}
\begin{proof}
\begin{align}
    
\end{align}
\begin{flushright}
$\blacksquare$
\end{flushright}
\end{proof}

\subsubsection{\textbf{$\left\lvert x - y \right\rvert \leq \left\lvert x \right\rvert + \left\lvert y \right\rvert$. (Give a very short proof.)}}
\begin{proof}
\begin{align}
    
\end{align}
\begin{flushright}
$\blacksquare$
\end{flushright}
\end{proof}

\subsubsection{\textbf{$\left\lvert x \right\rvert - \left\lvert y \right\rvert \leq \left\lvert x - y \right\rvert$. (A very short proof is possible, if you write things in the right way.)}}
\begin{proof}
\begin{align}
    
\end{align}
\begin{flushright}
$\blacksquare$
\end{flushright}
\end{proof}

\subsubsection{\textbf{$\left\lvert (\left\lvert x \right\rvert - \left\lvert y \right\rvert)\right\rvert \leq \left\lvert x - y \right\rvert$. (Why does this follow immediately from $(v)$?)}}
\begin{proof}
\begin{align}
    
\end{align}
\begin{flushright}
$\blacksquare$
\end{flushright}
\end{proof}

\subsubsection{\textbf{$\left\lvert x + y + z \right\rvert \leq \left\lvert x \right\rvert + \left\lvert y \right\rvert + \left\lvert z \right\rvert$. Indicate when equality holds, and prove your statement.}}
\begin{proof}
\begin{align}
    
\end{align}
\begin{flushright}
$\blacksquare$
\end{flushright}
\end{proof}

%%%%%%%%%%%%%%%%%%%%%%%%%%%%%%%%%%%%%%%%%%%%%%%%%%%%%%
%%%%%%%%%%%%%%%%%%% ACKNOWLEDGMENT %%%%%%%%%%%%%%%%%%%
%%%%%%%%%%%%%%%%%%%%%%%%%%%%%%%%%%%%%%%%%%%%%%%%%%%%%%
\newpage

\section*{ACKNOWLEDGMENT}
I would like to thank so much ...

\noindent\rule{8cm}{0.4pt}

%%%%%%%%%%%%%%%%%%%%%%%%%%%%%%%%%%%%%%%%%%%%%%%%%%%%%%
%%%%%%%%%%%%%%%%%%% References %%%%%%%%%%%%%%%%%%%%%%%
%%%%%%%%%%%%%%%%%%%%%%%%%%%%%%%%%%%%%%%%%%%%%%%%%%%%%%
\begin{thebibliography}{99}

\bibitem{c1} ETOIX, CALCULUS BY SPIVAK, CHAPTER 1, PROBLEM 6, SEPTEMBER 30, 2014 \href{https://etoix.wordpress.com/2014/09/30/calculus-by-spivak-chapter-1-problem-6/}{\underline{link}}
% \bibitem{c2} EthHub, ZK-STARKs. \href{https://docs.ethhub.io/ethereum-roadmap/layer-2-scaling/zk-starks/}{\underline{link}}
% \bibitem{c3} github.com/ebfull/powersoftau. \href{https://github.com/ebfull/powersoftau}{\underline{link}}
% \bibitem{c4} Koh Wei Jie, Announcing the Perpetual Powers of Tau Ceremony to benefit all zk-SNARK projects, Sep 11, 2019. \href{https://medium.com/coinmonks/announcing-the-perpetual-powers-of-tau-ceremony-to-benefit-all-zk-snark-projects-c3da86af8377#:~:text=The%20Powers%20of%20Tau%20ceremony,protocol%20can%20be%20publicly%20verified.&text=Nevertheless%2C%20each%20ceremony%20takes%20time%20and%20is%20tedious%20to%20coordinate.}{\underline{link}}
% \bibitem{c5} Polygon Hermez, Hermez Zero-Knowledge Proofs, 2020. \href{https://blog.hermez.io/hermez-zero-knowledge-proofs/}{\underline{link}}
% \bibitem{c6} Matthew Finestone, Loopring Begins zkSNARK Trusted Setup Multi-Party Computation Ceremony, Nov 10, 2019 \href{https://medium.com/coinmonks/loopring-starts-zksnark-trusted-setup-multi-party-computation-ceremony-cd1b98113773#:~:text=There%20are%20two%20phases%20for,circuit%20in%20the%20Loopring%20protocol.}{\underline{link}}
% \bibitem{c7} Circom official documentations. \href{https://docs.circom.io/circom-language/constraint-generation/}{\underline{link}}
% \bibitem{c8} Snarkjs official documentation \href{https://github.com/iden3/snarkjs/blob/master/README.md}{\underline{link}}
\end{thebibliography}

\end{document}
